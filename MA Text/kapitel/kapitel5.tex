% kapitel2.tex
\chapter{Evaluation}
\label{chapter:evaluation}

%
%\begin{figure}[!ht]
%\centering
%\begin{tikzpicture}
%	\begin{axis}[
%		title={Ant Colony Cases},
%		xlabel={Number of Ants},
%		ylabel={Covered Area},
%		width=0.9\textwidth, % Set the width of the plot
%		enlargelimits=false,
%		scatter/classes={%
%			blue={mark=*,blue},
%			% Add more colors if necessary
%		}
%		]
%		\pgfplotstableread[col sep=comma]{./data/antColonyCasesNumberAntsCovereArea.csv}\datatable
%		
%		% Plot data, assuming 'CoveredArea' and 'run_name_ants_Importance' are the column names
%		\addplot[scatter,only marks, scatter src=explicit symbolic]table[x=Importance, y=CoveredArea, meta=color] {\datatable};
%		
%	\end{axis}
%\end{tikzpicture}
%\caption{Multiple Line-plots using Tikz}
%\label{fig:my_label}
%\end{figure}

\begin{figure}[!ht]
	\centering
	\begin{tikzpicture}
		\begin{axis}[
			title={Ant Colony Cases},
			xlabel={Importance},
			ylabel={Covered Area},
			width=0.9\textwidth, % Set the width of the plot
			enlargelimits=false,
			]
			% Read the CSV file
			\pgfplotstableread[col sep=comma]{./data/antColonyCasesNumberAntsCovereArea.csv}\datatable
			
			\addplot[
			scatter,
			only marks, mark options={
				draw=Color,
				fill=Color,
			},
			] table [
			x=Importance,
			y=CoveredArea
			] {\datatable};
%			\addplot [no markers] table [y={create col/linear regression={y=CoveredArea}}] {./data/antColonyCasesNumberAntsCovereArea.csv};
		\end{axis}
	\end{tikzpicture}
	\caption{Multiple Line-plots using Tikz}
	\label{fig:my_label3e}
\end{figure}
% kapitel2.tex
\chapter{Related Work}
\label{chapter:relatedWork}


Much research has been done for shortest path algorithms and their optimization, however, for the - more complicated \cite{gemsaEfficientComputationJogging2013} - problem of finding a round trip with additional conditions, not much work has been done yet.
While there are a few tools that can be used to calculate round trips, most of them only focus on cycling or create a very limited set of trips that do not satisfy the needs of most people, or both. 
Some examples for these tools are RouteLoops \footnote{https://www.routeloops.com/} and RouteYou \footnote{https://www.routeyou.com} which both do not allow for much customization of preferences. 

Adding new options for user inputs that enable a higher degree of customization can vastly improve the usability of a tool. 
The usefulness is not only determined by the implemented algorithms, but also by the interface, the data used, and the selection options presented to the user. 

As both RouteLoops and RouteYou are commercial programs, it was not possible to obtain the necessary details about any used algorithms, heuristics, meta-heuristics or even the language they used for programming these solutions.
All gathered information are collected from exploring the functionality of the two tools by hand and reading both the general information and the FAQ pages provided by the websites. 

\section{Tour4Me}
\label{sec:Tour4Me}

The tool which this thesis will be based off, Tour4Me\footnote{http://tour4me.cs.tu-dortmund.de/} \cite{buchinTour4MeFrameworkCustomized2022a}, incorporates many of these points in its web interface. 
It is possible to choose the preferred ground type as well as make selections about preferred route types.
Furthermore, the user can also mark certain types as unpreferrable (rather than just keeping them neutral or preferring them).
This allows for much more customization.
What the tool does not incorporate yet is the option to make selections about the preferred elevation or route complexity.
However, the route can be optimized for a circular route when using the covered area of the tour. 

%Currently, Tour4Me does not work for all starting points and or lengths selected by the user.
%It uses different approaches and some cannot calculate results.
%Sometimes, it works to allow for a longer computing time but sometimes, no route can be found at all.
%The goal is to add more options that will be able to return results for most or - optimally - all starting points, lengths and user customizations.

It implements a solution for the "touring problem", which is used to describe the task of finding appealing and ideally interesting roundtrips.
To achieve an optimal solution, two factors are taken into consideration.
First is the total profit, that can be collected within the given length restriction for the tour.
Second is an additional quality function that assures for a relatively round tour by maximizing the area that is surrounded by the created roundtrip.
Tour4Me presents a selection of four different algorithms to calculate the tour as well as some additional customization options.
The offered choices include a Greedy Selection approach, Integer Linear Programming, MinCost with Waypoints, a shortest paths variant, and Iterative Local Search \cite{buchinTour4MeFrameworkCustomized2022a}. 

The Greedy Selection is the simplest algorithm which only ensures that the chosen route is a roundtrip.
It builds it's path by iterating over the valid edges and picking the most profitable of these until the cycle is finished or no candidate is left.
A valid edge is determined by checking whether the start- and endpoint s can still be reached if that edge is picked next \cite{buchinTour4MeFrameworkCustomized2022a}.


For Integer Linear Programming, the touring problem must be stated in an appropriate form.
To do so, a single instance can be encoded as $\mathcal{I}(G, w, \pi, B, v_0)$, containing the Graph $G$, edge costs $w$, the profit function $\pi$, the budget (length restrictions) $B$ and the starting (and end-) point $v_0$.
Given this encoding, cycles $P=(v_0,...,v_i,...,v_0)$, which are always at most of length L, can be built.
For the current definition, a few additional variables an be introduced to encode whether or not an edge is part of a solution (and how many times it occurs), whether or not an edge is the k-th edge of the solution and whether or not a vertex is the k-th vertex of a solution. 
Using these, constraints can be built to describe the desired behavior of the algorithm \cite{buchinTour4MeFrameworkCustomized2022a}.

The MinCost algorithm needs the waypoints because it is typically meant to solve shortest path problems. 
Thus it would always choose not leaving the starting position without the added points. 
Even though this algorithm is not originally meant to solve roundtrip problems, it takes into account the cost and profits of edges to create a solution tour, which makes it more suited to the task than simple greedy search.
To create an optimized tour, the inefficiency of paths has to be measured. 
This is done by calculating the quotient of the edge costs and the profit the edge yields. 
Using this inefficiency, a ring of candidate points $R_s$ surrounding the start-point s can be calculated.
All points that are part of this ring have a shortest path distance of at most $\pi$.
From these, new rings $R_r$ with the same requirements can be calculated. 
The solution path is then obtained through intersecting the sets of all circles and selecting all those that intersect with $R_s$.
To ensure the highest profit tour is returned, all possible combinations are calculated and the optimum is returned\cite{buchinTour4MeFrameworkCustomized2022a}.

Building from this solution, the Iterative Local Search can be applied to improve the found tours.
From the returned roundtrip, the algorithm removes partial paths $P$ from the current best solution $S$ and tries to iteratively add new parts that improve the solution profit while always staying within the given budget ($B-w(\frac{S}{P})$).
Since searching for viable edges is performed using a depth first approach, bounding the maximum depth of this step can drastically speed up the algorithm.
To keep track of the added length and profit, two variables ($l$ and $p$ respectively) are introduced.
These start with an initial value of one and are raised by a single increment for each iteration.
$p$ is reset when the starting point is reached by the removal step.
$l$ is reset when the maximum length for the solution is reached.
The best solution is improved constantly until the user selected time limit is reached \cite{buchinTour4MeFrameworkCustomized2022a}.

\#TODO add more citations --> see Tour4Me paper

\section{Roundtrip paths}
\label{sec:Roudtrip}
As already stated above, existing tools leave out certain data like elevation or path types.
This impacts the quality of the created routes for users or even user groups. 
For example, people who prefer running with little to no elevation can end up with a route that takes them uphill through a park for half of the route.
Which still may be a good choice for other users - joggers who prefer more challenging routes or people who want to hike and enjoy ascending.
However, others could prefer running through the city over a park when the elevation matches their preferences better in the city.
For these users, the created route would be highly unfavorable, even though it matches other constraints for what is considered a nice roundtrip.
Therefore, it can be crucial to the usefulness of an app to give the user as many options to customize as possible. 


\subsection{RouteLoops \& RouteYou}
\label{subsec:routeLoopsrouteYou}
RouteLoops has two text fields for entering the starting point and the length of the trip.
Aside from that, no real customization is possible.
It does have a few features to show more information about the route like showing distance markers or elevation, however, these can not be used as inputs to get a route with - for example - as little elevation as possible.
Apparently it can also show route difficulty for the United States, however even when creating a route in the United States, no result was shown. 
RouteLoops also does not actually create loops but rather picks a route that has high value (for example with a river in a park) and lets the user run along that path, turn around at the end and run back the same way.

To crate a roundtrip, some "waypoints" are created. 
These can be removed or more can be added in when editing the tour.
Between the waypoints, it seems like a shortest path is tried 

RouteYou offers several different options that will return varying results, however, picking the same option again will also give different results every time.   
Here, the roundtrips are more round than with RouteLoops, but again, elevation or difficulty are not taken into account. 
Also, while both do offer the possibility to edit the returned roundtrip, this editing changes the length of the route arbitrarily.
Furthermore, it is not possible to specify directly what type of underground or surroundings etc. are preferred. 

\subsection{Computing Running Routes}
\label{subsec:runningRoutes}

The problem of calculating good running roundtrips is not new.
In addition to the commercial applications, there also are research papers on this subject.
One of these papers is "Efficient Computation of Jogging Routes" \cite{gemsaEfficientComputationJogging2013}, presents two ideas to handle the new routing problem which the authors labeled "Jogging Problem".
It is split up into two variants: 
One being the simple version, that only aims to build a cycle that contains the starting point s and has the desired length.
The other is a more complex version, that allows for some flexibility regarding the length of the final tour during optimization. 
Hence, it is named "Relaxed Jogging Problem". 
This relaxation allows to take more factors into account to also optimize for the resulting shape, the area surrounding the tour and/or the simplicity of the path \cite{gemsaEfficientComputationJogging2013}. 

The second problem is chosen as the one to optimize, since it enables the addition of other conditions than just the length of a tour.
For this, two different ideas are proposed.
The first approach - "Greedy Faces" - is based on extending previous cycles.
It starts with a cycle containing the starting point s that can be selected by the user. 
This roundtrip then can be extended to gradually approach the user specified length. 
The second algorithm was named "Partial Shortest Paths" and uses via-vertices.
These are a number of new points that can be connected with shortest paths.
When the via-vertices are connected with each other and the start, they form a roundtrip \cite{gemsaEfficientComputationJogging2013}.


For both algorithms, the authors measure the badness of paths, the number of edges that are shared as well as the number of turns.
The badness is used to take the additional constraints into account. 
To reduce the possibility of having a roundtrip which turns at the end and uses all paths twice, the shared edges have to be minimized.
The number of turns corresponds to the complexity of the tour and is measured by a percentage of a full u-turn \cite{gemsaEfficientComputationJogging2013}. 

\#TODO how often do I need to reference the paper?


\paragraph{Greedy Faces}

Greedy Faces is built from an already existing circle by extending it.
For this, blocks outside the tour that are adjacent to the current path are used.
The previous circle encloses the chosen block and thus extends the previous route. 
New blocks are picked until the desired length is reached.
To ensure only blocks that correspond to faces are picked, a preprocessing phase is introduced that identifies faces of the graph.
During this step, first, dead-ends are removed, so the resulting graph will be two-connected.
Faces then are defined by the edges that surround them. 
While identifying all faces, a dual graph $G^{\star} = (V^{\star},E^{\star})$ for $G = (V,E)$ is built as well.

The Greedy Faces algorithm then works on the dual graph $G^{\star}$, selects a face f from $V^{\star}$ which has a surrounding path that contains the starting point s. 
Then, a Breadth First Search Tree T is built, starting at f, until the desired length (a relaxed version $(1 + \varepsilon) L$) is exceeded.
The resulting tour will be a simple path iff all vertices in V without the ones in T are connected and contain s.
The final jogging path can be extracted by taking the cut edges between the tree T and the remaining vertices.
This always forms a circle and thus builds a roundtrip.

For building a path which optimizes all constraints, the three introduced measures for badness, number of shared edges and the number of turns are used.
The badness function is incorporated into a different force function which can assign positive and negative badness values to edges.
Furthermore, the force function uses the cost of the face and a vector $\vec{d} = \vec{p} - \vec{C}(f)$ which is built from the geometric center $\vec{C}(f)$ of a face to any point $\vec{p}$
 


After the tour has been created, it will be smoothed to reduce the complexity.


\paragraph{Partial Shortest Paths}





\paragraph{Computational Complexity}

Aside from introducing two methods to calculate roundtrip tours for running, this paper also presents a proof for the computational complexity of their Simple and Relaxed Jogging Problems.
The authors show NP hardness by reduction of Hamiltonian Cycle to the optimization problem corresponding to their original problems.

 \# TODO add the actual proof?

\# TODO other jogging route paper







\subsection{Computing Cycling Routes}
\label{subsec:cyclingRoutes}
\# TODO cycling paper Tour4Me

\# TODO other cycling paper










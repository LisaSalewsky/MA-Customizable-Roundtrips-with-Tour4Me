% kapitel2.tex



\section{Goal and Methodology}
\label{sec:goal}


The goal of this thesis is to create a usable application for computing running or cycling roundtrips of (almost) arbitrary length. 
Usable in this case means an app that can be used in real time, that produces results of the desired length and prioritizes paths according to the users' input. 
To achieve this goal, the thesis will be built on the already existing prototype Tour4Me and eventually add meta-heuristic approaches that have been deemed the most fitting for this purpose. 

First, an interface for testing the new approaches has to be built. This interface also needs an overlay for adding in user options like the length of the desired roundtrip, as well as other preference inputs. 
Based on this interface, different algorithms can be added and compared with each other to find the ones that will produce relatively good outputs.
However, there can be very different definitions of what makes a result good or high quality. 
An ideal algorithm would be fast, always generate a route and use all the users' preference inputs.
However, it is not possible to achieve all these goals with just one algorithm. 
Therefore, different approaches will be implemented and analyzed according to how well they fulfill the previously mentioned criteria. 

Some of the possible approaches include different implementations of genetic algorithms \cite{gendreau_handbook_2010}, of ant colony or anthill algorithms \cite{gendreau_handbook_2010, babaoglu_anthill_2002, wang_application_2014} as well as possible hybrid versions.
These hybrids can either be hybrids of one of the meta-heuristics with - for example - local search algorithms \cite{gendreau_handbook_2010, wang_application_2014} or hybrids of these two algorithms joined together.
Furthermore, if there is enough time left, it is also possible to include the new algorithms into already implemented ones to improve those.

When the best algorithms for this application have been determined, they will be integrated into the already existing Tour4Me application. The aim is for the app to calculate a high-quality tour for any typical roundtrip requests for running and cycling.

In addition to finding suitable algorithms that allow for fast and reliable computation of all typical roundtrips, working on the interface and data used also improves the usefulness of the app.
It can be equally important to improve the interface, add more options like elevation data, include more information (for example previously used routes) etc. 
There are several opportunities and options to improve the app not only by changing the used algorithms but also through adding user selection options and upgrading the GUI.
The extension of available inputs and sliders to better specify tour parameters is an alternative approach towards the goal of making Tour4Me more usable.
It is another option to put more work into improving the app aside from adding more or faster algorithms.




\section{Structure}
\label{sec:structure}



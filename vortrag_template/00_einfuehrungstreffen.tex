\documentclass[aspectratio=169,xcolor=dvipsnames, t]{beamer}

\usepackage{cp24}

%*************************************************************************************************************
\title[Customizable Roundtrips with Tour4Me]{Customizable Roundtrips with Tour4Me}
\subtitle{Meta-heuristic Approaches for Personalized Running and Cycling Routes}
\author[Lisa Salewsky]{
	Lisa Salewsky
	}


\institute[TU Dortmund]{
	 \begin{minipage}[t][2.3cm][b]{0.7\textwidth}
		\centering
		\begin{tikzpicture}[remember picture,overlay]
			\node[inner sep=0] at (2.5,-0.8) {\includegraphics[scale=0.1]{logo.png}};
		\end{tikzpicture}
	 	TU Dortmund, Fakultät für Informatik\\
		\vspace{1.5cm}
		Reviewer:\\
		Prof. Dr. Kevin Buchin\\
		Mart Hagedoorn, M. Sc.
	\end{minipage}
	
}
\date[\today]{\today}
\beamertemplatenavigationsymbolsempty %Navigantionszeite unten aus
%*************************************************************************************************************
%\usetheme{Malmoe} 
%\usetheme{Madrid}
%\usecolortheme{rose} %schwache farben
% Themes:
%AnnArbor | Antibes | Bergen |Berkeley | Berlin | Boadilla |boxes | CambridgeUS | Copenhagen |Darmstadt | default | Dresden |
%Frankfurt | Goettingen |Hannover | Ilmenau | JuanLesPins | Luebeck | Madrid | Malmoe | Marburg |
%Montpellier | PaloAlto | Pittsburgh |Rochester | Singapore | Szeged |Warsaw
\useinnertheme{rounded}
%\useoutertheme{shadow} %schattierung in den Rechtecken der Blocken -- nix
%*************************************************************************************************************


% Customize the table of contents
\setbeamertemplate{section in toc}[square]
\setbeamertemplate{subsection in toc}[square]
\setbeamertemplate{section/subsection in toc}[square]
\setbeamercolor{section number projected}{bg=tu_green_full,fg=white}
\setbeamercolor{subsection number projected}{bg=tu_green_full,fg=white}

% Customize all the bulletpoints
\setbeamertemplate{items}[square]

% Customize the link colors
\hypersetup{
	colorlinks=true,
	linkcolor=tu_green_full2,
	urlcolor=tu_green_full2,
	citecolor=tu_green_full2
}

\setbeamercolor{block title}{bg=tu_light_green_full2, fg=black}
\setbeamercolor{block body}{bg=tu_light_green_full3, fg=black}


\begin{document}
%*************************************************************
\begin{frame}
	\thispagestyle{empty}
	\titlepage
\end{frame}
%*************************************************************

\begin{frame}{Agenda}
	\centering	
	\tableofcontents
\end{frame}

\section{Einf\"uhrung}
%*************************************************************
\begin{frame}{Beispiel Box}
	\begin{block}{\href{https://icpc.global/}{International Collegiate Programming Contest (ICPC)}}
		\begin{itemize}
			\item Association for Computing Machinery (ACM)
			\item seit 1970
			\item an Universit\"uten weltweit
		\end{itemize}
	\end{block}
	\begin{itemize}
		\item Teams von 3 Studierenden
		\item 10 Probleme mit verschiedenem Schwierigkeitsgrad
		\item 1 Computer pro Gruppe
		\item Hilfsmittel: \glqq Cheat Sheet\grqq
		\item L\"osungen werden zu einem Judge Server hochgeladen
		\item Gewinner: die Gruppe, welche die meisten Probleme gel\"ost hat
	\end{itemize}

\end{frame}

\section{Organisatorisches}


%*************************************************************
\begin{frame}{Beispiel column plus Box}
	\vspace{-0.5cm}
	\begin{columns}
		\begin{column}{.65\linewidth}
			\begin{itemize}
				{\small
				\item { \bf 11.10.} Einf\"uhrungstreffen
				\item { \bf 18.10.} Systemeinf\"uhrung
				\item { \bf 25.10.} Tipps und Tricks
				\item { \bf 08.11.} Datenstrukturen und Algorithmenentwurfsmethoden
				\item { \bf 15.11.} Such- und Sortieralgorithmen
				\item { \bf 22.11.} Dynamisch Programmieren
				\item { \bf 29.11.} Strings
				\item { \bf 06.12.} \"Ubungswettbewerb 1
				\item { \bf 13.12.} Graphtraversierung
				\item { \bf 20.12.} Flussalgorithmen und Matchings
				\item { \bf 10.01.} Algorithmische Geometrie
				\item { \bf 17.01.} \"Ubungswettbewerb 2
				\item { \bf 24.01.} Wintercontest oder Interner Wettbewerb 
    		}
			\end{itemize}
		\end{column}
		\begin{column}{.25\linewidth}
			\begin{block}{W\"ochentliche Treffen}
				12:15 -- ca.\ 13:30\\
				Besprechung, Vortrag\\
				13:30 -- 15:45\\
				Probleme l\"osen, Hilfestellung
			\end{block}
		\end{column}
	\end{columns}

\end{frame}
%*************************************************************
\begin{frame}{Beispiel Spalten mit Boxen}
	\begin{columns}
		\begin{column}{0.3\textwidth}
			\begin{block}{Kommunikation}
				\begin{enumerate}
					\item Sprache
					\item Stimme \& K\"orpersprache
					\item Einfachheit \& Pr\"ogranz, Zeit
				\end{enumerate}
			\end{block}
		\end{column}
		\pause
		\begin{column}{0.3\textwidth}
			\begin{block}{Methodik}
				\begin{enumerate}
					\item Struktur
					\item Stimulanz
					\item Medien- \& Materialeinsatz
					\item Interaktion
				\end{enumerate}
			\end{block}
		\end{column}
		\pause
		\begin{column}{0.3\textwidth}
			\begin{block}{Vortragsqualit\"at}
				\begin{enumerate}
					\item Korrektheit \& Technische Tiefe
					\item Beantowrtung der Fragen
					\item Pr\"asentationsziel
				\end{enumerate}
			\end{block}
		\end{column}
	\end{columns} ~\\
	Motivation beim Vortrag!
\end{frame}

\begin{frame}{Timeline}
   \centering
   \vspace{0.5cm}
   \tikzstyle{descript} = [text = black,align=center, minimum height=1.8cm, align=center, outer sep=0pt,font = \footnotesize]
\begin{tikzpicture}[very thick, black]
	\small
	
	%% Coordinates
	\coordinate (O) at (-1,0); % Origin
	\coordinate (P1) at (1,0);
	\coordinate (P2) at (8,0);
	\coordinate (P3) at (11,0);
	\coordinate (P4) at (12.5,0);
	\coordinate (F) at (13,0); %End
	\coordinate (E1) at (5,0); %Event
	\coordinate (E2) at (0.5,0); %Event
	
		
	%% Arrow
	\draw[->] (O) -- (F);
	%% Ticks
	\foreach \x in {0,2,...,12}
	\draw(\x cm,3pt) -- (\x cm,-3pt);
	%% Labels
	\foreach \i \j in {0/Nov,2/Dec,4/Jan,6/Feb,8/Mar,10/Apr,12/May}{
		\draw (\i,0) node[below=3pt] {\j} ;
	}
	\pause
	
	%% Filled regions
	\fill[color=ColorOne!20] rectangle ($(O)+(0.2,0)$) -- (P1) -- ($(P1)+(0,1)$) -- ($(O)+(0.2,1)$);
	\path [pattern color=ColorOne!80, pattern=north east lines, line width = 1pt, very thick] rectangle ($(O)+(0.4,0)$) -- ($(O)+(2,0)$) -- ($(O)+(2,1)$) -- ($(O)+(0.4,1)$);
	\draw ($(P1)+(-0.9,0.5)$) node[activity,ColorOne] {Pre-work};
	\pause	
	
	%% Description
	\node[descript,fill=ColorOne!15,text=ColorOne](P0) at ($(P1)+(2,-2.5)$) {%
		literature\\
		examples\\
		implementing interface};
	
	%% Arrows
	\path[->,color=ColorOne] ($(P1)+(-0.5,-0.1)$) edge [out=-90, in=130]  ($(P0)+(0,1)$);
	\pause
	
	\fill[color=ColorTwo!15] rectangle (P1) -- (P2) -- ($(P2)+(0,1)$) -- ($(P1)+(0,1)$);
	\draw ($(P2)+(-3.55,0.5)$) node[activity,ColorTwo] {Implementing and testing different algorithms,\\ changing other aspects of the app};
	\pause
	
	\fill[ColorThree!30] rectangle (P2) -- (P3) -- ($(P3)+(0,1)$) -- ($(P2)+(0,1)$);
	\draw ($(P3)+(-1.55,0.5)$)  node[activity, ColorThree] {Gathering \\results};
	\pause
	
	\fill[ColorFour!20] rectangle (P3) -- (P4) -- ($(P4)+(0,1)$) -- ($(P3)+(0,1)$);
	\draw ($(P4)+(-0.8,0.5)$)  node[activity, ColorFour!80!black] {Writing \\only};
	\pause

	
	%% Events
	\draw [decorate,decoration={brace,amplitude=6pt}]($(P1)+(-1.8,1.2)$) -- ($(F)+(-0.5,1.2)$) node [black,midway,above=6pt, align=center] {Writing on the thesis};

	
\end{tikzpicture}
\end{frame}
%*************************************************************
\begin{frame}{Thema 1: Datenstrukturen und Algorithmenentwurfsmethoden}
	\begin{columns}
		\begin{column}{0.5\textwidth}
			\begin{itemize}
				\item Listen
				\item Arrays
				\item Stacks
				\item Heaps
				\item Hashing
			\end{itemize}
		\end{column}
		% \end{frame}
		% %*************************************************************
		% \begin{frame}{Thema 2: }
		\begin{column}{0.5\textwidth}
			\begin{itemize}
				\item Greedy
				\item Divide \& Conquer
				\item Brute-Force
				\item Backtracking
			\end{itemize}
		\end{column}
	\end{columns}
\end{frame}

%*************************************************************

%*************************************************************
%*************************************************************
\end{document}

